\documentclass{article}
\usepackage[utf8]{inputenc}
\usepackage[margin=1in]{geometry} 

\title{Algebra II Planning Notes}
\author{Steven Taylor}
\date{June 22, 2019}

\begin{document}

\maketitle
\section*{General Notes}
\begin{enumerate}
    \item OER Bookdown Temple [Complete - June 22, 2019]
    \item CCSS Alignment [Complete  - June 24, 2019]
    \begin{enumerate}
    	\item Determine Essential, Supporting, Advanced, and Advanced Supporting Standards [Complete - June 23, 2019]
    	\item Group Standards by Topic and or Unit [Complete - June 24, 2019]
    \end{enumerate}
    \item Learning Targets [First Pass Complete - June 24, 2019, Need to Prune]
    \item Scope \& Sequence
    \item Learning Target Quizzes
    \item Guided Practice Exercises
    \item Challenge Problems
\end{enumerate}
\section*{CCSS Alignment}
\subsection*{Essential Standards}
\begin{enumerate}
	\item CC.9-12.A.APR.1  Understand that polynomials form a system analogous to the integers, namely, they are closed under the operations of addition, subtraction, and multiplication; add, subtract, and multiply polynomials.
	\item CC.9-12.A.APR.3  Identify zeros of polynomials when suitable factorizations are available, and use the zeros to construct a rough graph of the function defined by the polynomial.
	\item CC.9-12.A.CED.1  Create equations and inequalities in one variable and use them to solve problems. 
	\item CC.9-12.A.CED.2  Create equations in two or more variables to represent relationships between quantities; graph equations on coordinate axes with labels and scales.
	\item CC.9-12.A.CED.4  Rearrange formulas to highlight a quantity of interest, using the same reasoning as in solving equations. 
	\item CC.9-12.A.REI.2 Solve simple rational and radical equations in one variable, and give examples showing how extraneous solutions may arise.
	\item CC.9-12.A.REI.4a Use the method of completing the square to transform any quadratic equation in x into an equation of the form $(x – p)^2 = q$ that has the same solutions. Derive the quadratic formula from this form. 
	\item CC.9-12.A.REI.4b Solve quadratic equations by inspection (e.g., for $x^2 = 49$), taking square roots, completing the square, the quadratic formula and factoring, as appropriate to the initial form of the equation. Recognize when the quadratic formula gives complex solutions and write them as $a \pm b\textrm{i}$ for real numbers $a$ and $b$.
	\item CC.9-12.A.REI.7 . Solve a simple system consisting of a linear equation and a quadratic equation in two variables algebraically and graphically. 
	\item CC.9-12.A.REI.11  Explain why the x-coordinates of the points where the graphs of the equations $y = f(x)$ and $y = g(x)$ intersect are the solutions of the equation $f(x) = g(x)$; find the solutions approximately, e.g., using technology to graph the functions, make tables of values, or find successive approximations. Include cases where f(x) and/or g(x) are linear, polynomial, rational, absolute value, exponential, and logarithmic functions.
	\item CC.9-12.A.SSE.3 Choose and produce an equivalent form of an expression to reveal and explain properties of the quantity represented by the expression.
	\item CC.9-12.A.SSE.3a Factor a quadratic expression to reveal the zeros of the function it defines.
	\item CC.9-12.A.SSE.3b Complete the square in a quadratic expression to reveal the maximum or minimum value of the function it defines.*
	\item CC.9-12.A.SSE.3c Use the properties of exponents to transform expressions for exponential functions.
	\item CC.9-12.F.BF.2 Build a function that models a relationship between two quantities. Write arithmetic and geometric sequences both recursively and with an explicit formula, use them to model situations, and translate between the two forms.
	\item CC.9-12.F.BF.3 Build new functions from existing functions. Identify the effect on the graph of replacing $f(x)$ by $f(x) + k, k f(x), f(kx)$, and $f(x + k)$ for specific values of k (both positive and negative); find the value of k given the graphs. Experiment with cases and illustrate an explanation of the effects on the graph using technology. Include recognizing even and odd functions from their graphs and algebraic expressions for them.
	\item CC.9-12.F.BF.4  Find inverse functions. 
	\item CC.9-12.F.BF.4a Solve an equation of the form $f(x) = c$ for a simple function f that has an inverse and write an expression for the inverse. 
	\item CC.9-12.F.IF.4 For a function that models a relationship between two quantities, interpret key features of graphs and tables in terms of the quantities, and sketch graphs showing key features given a verbal description of the relationship. 
	\item CC.9-12.F.IF.7 Graph functions expressed symbolically and show key features of the graph, by hand in simple cases and using technology for more complicated cases.
	\item CC.9-12.F.IF.7c Graph polynomial functions, identifying zeros when suitable factorizations are available, and showing end behavior.
	\item CC.9-12.F.IF.7e Graph exponential and logarithmic functions, showing intercepts and end behavior, and trigonometric functions, showing period, midline, and amplitude.
	\item CC.9-12.F.LE.1  Distinguish between situations that can be modeled with linear functions and with exponential functions.
	\item CC.9-12.F.LE.1a Prove that linear functions grow by equal differences over equal intervals and that exponential functions grow by equal factors over equal intervals.
	\item CC.9-12.F.LE.1b. Recognize situations in which one quantity changes at a constant rate per unit interval relative to another.
	\item CC.9-12.F.LE.1c Recognize situations in which a quantity grows or decays by a constant percent rate per unit interval relative to another.
	\item CC.9-12.F.LE.4 For exponential models, express as a logarithm the solution to$ ab^{ct} = d$ where $a, c$, and $d$ are numbers and the base $b$ is 2, 10, or e; evaluate the logarithm using technology.
	\item CC.9-12.F.LE.5  Interpret the parameters in a linear or exponential function in terms of a context.
	\item CC.9-12.F.TF.2  Explain how the unit circle in the coordinate plane enables the extension of trigonometric functions to all real numbers, interpreted as radian measures of angles traversed counterclockwise around the unit circle.
	\item CC.9-12.F.TF.5  Choose trigonometric functions to model periodic phenomena with specified amplitude, frequency, and midline.
	\item CC.9-12.F.TF.8  Prove the Pythagorean identity $(\sin A)^2 + (\cos A)^2 = 1$ and use it to find $\sin A, \cos A$, or $\tan A$, given $\sin A, \cos A$, or$\ tan A$, and the quadrant of the angle.
	\item CC.9-12.G.GPE.4 Use coordinates to prove simple geometric theorems algebraically. 
	\item CC.9-12.G.SRT.8 . Use trigonometric ratios and the Pythagorean Theorem to solve right triangles in applied problems.
	\item CC.9-12.N.CN.2 Use the relation $\textrm{i}^2 = –1$ and the commutative, associative, and distributive properties to add, subtract, and multiply complex numbers.
	\item CC.9-12.N.CN.7  Solve quadratic equations with real coefficients that have complex solutions.
	\item CC.9-12.N.Q.1  Use units as a way to understand problems and to guide the solution of multi-step problems; choose and interpret units consistently in formulas; choose and interpret the scale and the origin in graphs and data displays.
	\item CC.9-12.N.RN.2  Rewrite expressions involving radicals and rational exponents using the properties of exponents.
	\item CC.9-12.S.ID.4  Use the mean and standard deviation of a data set to fit it to a normal distribution and to estimate population percentages. Recognize that there are data sets for which such a procedure is not appropriate. Use calculators, spreadsheets, and tables to estimate areas under the normal curve.
	\item CC.9-12.S.ID.6  Represent data on two quantitative variables on a scatter plot, and describe how the variables are related.
	\item CC.9-12.S.ID.6a Fit a function to the data; use functions fitted to data to solve problems in the context of the data. 
\end{enumerate}
\subsection*{Supporting Standards}
\begin{enumerate}
	\item CC.9-12.A.APR.2 Know and apply the Remainder Theorem: For a polynomial $p(x)$ and a number $a$, the remainder on division by $x – a$ is $p(a)$, so $p(a) = 0$ if and only if $(x – a)$ is a factor of $p(x)$.
	\item CC.9-12.A.APR.4  Prove polynomial identities and use them to describe numerical relationships. For example, the polynomial identity $(x^2 + y^2)^2 = (x^2 – y^2)^2 + (2xy)^2$ can be used to generate Pythagorean triples.
	\item CC.9-12.A.APR.6 Rewrite rational expressions. Rewrite simple rational expressions in different forms; write $a(x)/b(x)$ in the form $q(x) + r(x)/b(x)$, where $a(x), b(x), q(x)$, and $r(x)$ are polynomials with the degree of $r(x)$ less than the degree of $b(x)$, using inspection, long division, or, for the more complicated examples, a computer algebra system.
	\item CC.9-12.A.CED.3  Represent constraints by equations or inequalities, and by systems of equations and/or inequalities, and interpret solutions as viable or non-viable options in a modeling context. 
	\item CC.9-12.A.REI.1   Explain each step in solving a simple equation as following from the equality of numbers asserted at the previous step, starting from the assumption that the original equation has a solution. Construct a viable argument to justify a solution method.
	\item CC.9-12.A.REI.10  Understand that the graph of an equation in two variables is the set of all its solutions plotted in the coordinate plane, often forming a curve (which could be a line).
	\item CC.9-12.A.SSE.1  Interpret expressions that represent a quantity in terms of its context.
	\item CC.9-12.A.SSE.1a Interpret parts of an expression, such as terms, factors, and coefficients.*
	\item CC.9-12.A.SSE.1b Interpret complicated expressions by viewing one or more of their parts as a single entity. 
	\item CC.9-12.A.SSE.2  Use the structure of an expression to identify ways to rewrite it. 
	\item CC.9-12.A.SSE.4  Derive the formula for the sum of a finite geometric series (when the common ratio is not 1), and use the formula to solve problems. 
	\item CC.9-12.F.BF.1 Write a function that describes a relationship between two quantities.
	\item CC.9-12.F.BF.1a Determine an explicit expression, a recursive process, or steps for calculation from a context. 
	\item CC.9-12.F.BF.1b Combine standard function types using arithmetic operations. 
	\item CC.9-12.F.IF.3 . Recognize that sequences are functions, sometimes defined recursively, whose domain is a subset of the integers. 
	\item CC.9-12.F.IF.5  Relate the domain of a function to its graph and, where applicable, to the quantitative relationship it describes.
	\item CC.9-12.F.IF.6   Calculate and interpret the average rate of change of a function (presented symbolically or as a table) over a specified interval. Estimate the rate of change from a graph.
	\item CC.9-12.F.IF.8  Write a function defined by an expression in different but equivalent forms to reveal and explain different properties of the function. 
	\item CC.9-12.F.IF.8a Use the process of factoring and completing the square in a quadratic function to show zeros, extreme values, and symmetry of the graph, and interpret these in terms of a context. 
	\item CC.9-12.F.IF.8b Use the properties of exponents to interpret expressions for exponential functions. 
	\item CC.9-12.F.IF.9 Compare properties of two functions each represented in a different way (algebraically, graphically, numerically in tables, or by verbal descriptions).
	\item CC.9-12.F.LE.2  Construct linear and exponential functions, including arithmetic and geometric sequences, given a graph, a description of a relationship, or two input-output pairs (include reading these from a table).
	\item CC.9-12.F.LE.3  Observe using graphs and tables that a quantity increasing exponentially eventually exceeds a quantity increasing linearly, quadratically, or (more generally) as a polynomial function.
	\item CC.9-12.F.TF.1 Understand radian measure of an angle as the length of the arc on the unit circle subtended by the angle.
	\item CC.9-12.G.C.5  Derive using similarity the fact that the length of the arc intercepted by an angle is proportional to the radius, and define the radian measure of the angle as the constant of proportionality; derive the formula for the area of a sector.
	\item CC.9-12.G.GPE.1  Derive the equation of a circle of given center and radius using the Pythagorean Theorem; complete the square to find the center and radius of a circle given by an equation.
	\item CC.9-12.G.GPE.2 Derive the equation of a parabola given a focus and directrix.
	\item CC.9-12.G.GPE.7  Use coordinates to compute perimeters of polygons and areas of triangles and rectangles, e.g., using the distance formula.
	\item CC.9-12.G.SRT.6 Understand that by similarity, side ratios in right triangles are properties of the angles in the triangle, leading to definitions of trigonometric ratios for acute angles.
	\item CC.9-12.G.SRT.7  Explain and use the relationship between the sine and cosine of complementary angles.
	\item CC.9-12.N.CN.1 Know there is a complex number i such that $\textrm{i}^2 =-1$, and every complex number has the form $a + b\textrm{i}$ with $a$ and $b$ real.  
	\item CC.9-12.N.RN.1 Explain how the definition of the meaning of rational exponents follows from extending the properties of integer exponents to those values, allowing for a notation for radicals in terms of rational exponents. 
	\item CC.9-12.S.IC.1  Understand statistics as a process for making inferences about population parameters based on a random sample from that population.
	\item CC.9-12.S.IC.2  Decide if a specified model is consistent with results from a given data-generating process, e.g., using simulation. 
	\item CC.9-12.S.IC.3  Recognize the purposes of and differences among sample surveys, experiments, and observational studies; explain how randomization relates to each.
	\item CC.9-12.S.IC.4  Use data from a sample survey to estimate a population mean or proportion; develop a margin of error through the use of simulation models for random sampling.
	\item CC.9-12.S.IC.5  Use data from a randomized experiment to compare two treatments; use simulations to decide if differences between parameters are significant.
	\item CC.9-12.S.IC.6  Evaluate reports based on data.
	\item CC.9-12.S.ID.3 Interpret differences in shape, center, and spread in the context of the data sets, accounting for possible effects of extreme data points (outliers).
	\item CC.9-12.S.ID.8  Compute (using technology) and interpret the correlation coefficient of a linear fit.
	\item CC.9-12.S.ID.9  Distinguish between correlation and causation.
\end{enumerate}
\subsection*{Advanced Standards}
\begin{enumerate}
	\item CC.9-12.F.BF.1c (+) Compose functions. 
	\item CC.9-12.F.BF.4b (+) Verify by composition that one function is the inverse of another. 
	\item CC.9-12.F.BF.4c (+) Read values of an inverse function from a graph or a table, given that the function has an inverse.
	\item CC.9-12.F.BF.4d (+) Produce an invertible function from a non-invertible function by restricting the domain.
	\item CC.9-12.F.IF.7d (+) Graph rational functions, identifying zeros and asymptotes when suitable factorizations are available, and showing end behavior. 
\end{enumerate}
\subsection*{Advanced Supporting Standards}
\begin{enumerate}
	\item CC.9-12.A.APR.5 (+) Know and apply that the Binomial Theorem gives the expansion of $(x + y)^n$ in powers of $x$ and $y$ for a positive integer $n$, where $x$ and $y$ are any numbers, with coefficients determined for example by Pascal’s Triangle. 
	\item CC.9-12.A.APR.7 (+)  Understand that rational expressions form a system analogous to the rational numbers, closed under addition, subtraction, multiplication, and division by a nonzero rational expression; add, subtract, multiply, and divide rational expressions.
	\item CC.9-12.A.REI.8 (+)  Represent a system of linear equations as a single matrix equation in a vector variable.
	\item CC.9-12.A.REI.9 (+)  Find the inverse of a matrix if it exists and use it to solve systems of linear equations (using technology for matrices of dimension $3 \times 3$ or greater).
	\item CC.9-12.F.BF.5 (+) Understand the inverse relationship between exponents and logarithms and use this relationship to solve problems involving logarithms and exponents.
	\item CC.9-12.F.TF.3 (+)  Use special triangles to determine geometrically the values of sine, cosine, tangent for $\pi3, \pi/4$ and $\pi/6$, and use the unit circle to express the values of sine, cosine, and tangent for $\pi - x, \pi + x$, and $2\pi - x$ in terms of their values for $x$, where $x$ is any real number.
	\item CC.9-12.F.TF.4 (+)  Use the unit circle to explain symmetry (odd and even) and periodicity of trigonometric functions.
	\item CC.9-12.F.TF.6 (+)  Understand that restricting a trigonometric function to a domain on which it is always increasing or always decreasing allows its inverse to be constructed.
	\item CC.9-12.F.TF.7 (+) Use inverse functions to solve trigonometric equations that arise in modeling contexts; evaluate the solutions using technology, and interpret them in terms of the context.*
	\item CC.9-12.F.TF.9 (+)  Prove the addition and subtraction formulas for sine, cosine, and tangent and use them to solve problems.
	\item CC.9-12.G.GPE.3 (+)  Derive the equations of ellipses and hyperbolas given the foci, using the fact that the sum or difference of distances from the foci is constant.
	\item CC.9-12.G.SRT.9 (+)   Derive the formula $A = (1/2)ab \sin(C)$ for the area of a triangle by drawing an auxiliary line from a vertex perpendicular to the opposite side.
	\item CC.9-12.G.SRT.10 (+) Prove the Laws of Sines and Cosines and use them to solve problems.
	\item CC.9-12.G.SRT.11 (+)  Understand and apply the Law of Sines and the Law of Cosines to find unknown measurements in right and non-right triangles (e.g., surveying problems, resultant forces).
	\item CC.9-12.N.CN.3 (+)  Find the conjugate of a complex number; use conjugates to find moduli and quotients of complex numbers.
	\item CC.9-12.N.CN.4 (+) Represent complex numbers on the complex plane in rectangular and polar form (including real and imaginary numbers), and explain why the rectangular and polar forms of a given complex number represent the same number.
	\item CC.9-12.N.CN.5 (+) Represent addition, subtraction, multiplication, and conjugation of complex numbers geometrically on the complex plane; use properties of this representation for computation. 
	\item CC.9-12.N.CN.6 (+)  Calculate the distance between numbers in the complex plane as the modulus of the difference, and the midpoint of a segment as the average of the numbers at its endpoints.
	\item CC.9-12.N.CN.8 (+)  Extend polynomial identities to the complex numbers.
	\item CC.9-12.N.CN.9 (+)  Know the Fundamental Theorem of Algebra; show that it is true for quadratic polynomials.
	\item CC.9-12.N.VM.1 (+) Recognize vector quantities as having both magnitude and direction. Represent vector quantities by directed line segments, and use appropriate symbols for vectors and their magnitudes (e.g., \textbf{$v$}, $|$\textbf{$v$}$|$, $||$\textbf{$v$}$||$, $\vec{v}$).
	\item CC.9-12.N.VM.2 (+)  Find the components of a vector by subtracting the coordinates of an initial point from the coordinates of a terminal point.
	\item CC.9-12.N.VM.3 (+)   Solve problems involving velocity and other quantities that can be represented by vectors.
	\item CC.9-12.N.VM.4 (+)  Add and subtract vectors.
	\item CC.9-12.N.VM.4a (+) Add vectors end-to-end, component-wise, and by the parallelogram rule. Understand that the magnitude of a sum of two vectors is typically not the sum of the magnitudes.
	\item CC.9-12.N.VM.4b (+) Given two vectors in magnitude and direction form, determine the magnitude and direction of their sum. 
	\item CC.9-12.N.VM.4c (+) Understand vector subtraction $v – w$ as $v + (–w)$, where $(–w)$ is the additive inverse of $w$, with the same magnitude as w and pointing in the opposite direction. Represent vector subtraction graphically by connecting the tips in the appropriate order, and perform vector subtraction component-wise.
	\item CC.9-12.N.VM.5 (+)  Multiply a vector by a scalar.
	\item CC.9-12.N.VM.5a (+) Represent scalar multiplication graphically by scaling vectors and possibly reversing their direction; perform scalar multiplication component-wise, e.g., as $c(\vec{v}_x, \vec{v}_y) = (c\vec{v}_x, c\vec{v}_y)$.
	\item CC.9-12.N.VM.5b (+) Compute the magnitude of a scalar multiple $c\vec{v}$ using $||c\vec{v}|| =|c|\vec{v}$. Compute the direction of $c\vec{v}$ knowing that when $|c|\vec{v} \neq 0$, the direction of $c\vec{v}$ is either along $\vec{v}$ (for $c > 0$) or against $\vec{v}$ (for $c < 0$).
	\item CC.9-12.N.VM.6 (+)  Use matrices to represent and manipulate data, e.g., to represent payoffs or incidence relationships in a network. 
	\item CC.9-12.N.VM.7 (+)  Multiply matrices by scalars to produce new matrices, e.g., as when all of the payoffs in a game are doubled. 
	\item CC.9-12.N.VM.8 (+)  Add, subtract, and multiply matrices of appropriate dimensions.
	\item CC.9-12.N.VM.9 (+) Understand that, unlike multiplication of numbers, matrix multiplication for square matrices is not a commutative operation, but still satisfies the associative and distributive properties. 
	\item CC.9-12.N.VM.10 (+) Understand that the zero and identity matrices play a role in matrix addition and multiplication similar to the role of 0 and 1 in the real numbers. The determinant of a square matrix is nonzero if and only if the matrix has a multiplicative inverse. 
	\item CC.9-12.N.VM.11 (+) Multiply a vector (regarded as a matrix with one column) by a matrix of suitable dimensions to produce another vector. Work with matrices as transformations of vectors.
	\item CC.9-12.N.VM.12 (+) Work with $2 \times 2$ matrices as transformations of the plane, and interpret the absolute value of the determinant in terms of area. 
	\item CC.9-12.S.CP.8 (+)  Apply the general Multiplication Rule in a uniform probability model, $\textrm{P}(A \textrm{and} B) = [\textrm{P}(A)]\times[\textrm{P}(B|A)] =[\textrm{P}(B)]\times[\textrm{P}(A|B)]$, and interpret the answer in terms of the model.
	\item CC.9-12.S.CP.9 (+) Use permutations and combinations to compute probabilities of compound events and solve problems.
	\item CC.9-12.S.MD.1 (+) Define a random variable for a quantity of interest by assigning a numerical value to each event in a sample space; graph the corresponding probability distribution using the same graphical displays as for data distributions.
	\item CC.9-12.S.MD.2 (+) Calculate the expected value of a random variable; interpret it as the mean of the probability distribution.
	\item CC.9-12.S.MD.3 (+) Develop a probability distribution for a random variable defined for a sample space in which theoretical probabilities can be calculated; find the expected value. 
	\item CC.9-12.S.MD.4 (+)  Develop a probability distribution for a random variable defined for a sample space in which probabilities are assigned empirically; find the expected value. 
	\item CC.9-12.S.MD.5 (+)  Weigh the possible outcomes of a decision by assigning probabilities to payoff values and finding expected values.
	\item CC.9-12.S.MD.5a (+) Find the expected payoff for a game of chance. 
	\item CC.9-12.S.MD.5b (+)  Evaluate and compare strategies on the basis of expected values. 
	\item CC.9-12.S.MD.6 (+)  Use probabilities to make fair decisions (e.g., drawing by lots, using a random number generator).
	\item CC.9-12.S.MD.7 (+)  Analyze decisions and strategies using probability concepts (e.g., product testing, medical testing, pulling a hockey goalie at the end of a game).
\end{enumerate}
\newpage
\section*{Standards Grouped by Topic}
\subsection*{General}
\begin{enumerate}
	\item Essential
	\begin{itemize}
		\item CC.9-12.A.CED.1  Create equations and inequalities in one variable and use them to solve problems. 
		\item CC.9-12.A.CED.2  Create equations in two or more variables to represent relationships between quantities; graph equations on coordinate axes with labels and scales.
		\item CC.9-12.A.CED.4  Rearrange formulas to highlight a quantity of interest, using the same reasoning as in solving equations. 
		\item CC.9-12.N.Q.1  Use units as a way to understand problems and to guide the solution of multi-step problems; choose and interpret units consistently in formulas; choose and interpret the scale and the origin in graphs and data displays.
	\end{itemize}		
	\item Supporting
	\begin{itemize}
		\item CC.9-12.A.CED.3  Represent constraints by equations or inequalities, and by systems of equations and/or inequalities, and interpret solutions as viable or non-viable options in a modeling context. 
		\item CC.9-12.A.REI.1   Explain each step in solving a simple equation as following from the equality of numbers asserted at the previous step, starting from the assumption that the original equation has a solution. Construct a viable argument to justify a solution method.
		\item CC.9-12.A.REI.10  Understand that the graph of an equation in two variables is the set of all its solutions plotted in the coordinate plane, often forming a curve (which could be a line).
		\item CC.9-12.A.SSE.1  Interpret expressions that represent a quantity in terms of its context.
		\item CC.9-12.A.SSE.1b Interpret complicated expressions by viewing one or more of their parts as a single entity.
		\item CC.9-12.A.SSE.2  Use the structure of an expression to identify ways to rewrite it.
	\end{itemize}
	\item Advanced
	\begin{itemize}
		\item
	\end{itemize}
	\item Advanced Supporting
	\begin{itemize}
		\item
	\end{itemize}
\end{enumerate}
\newpage
\subsection*{Standards - Functions}
\begin{enumerate}
	\item Essential
	\begin{itemize}
		\item CC.9-12.A.REI.11  Explain why the x-coordinates of the points where the graphs of the equations $y = f(x)$ and $y = g(x)$ intersect are the solutions of the equation $f(x) = g(x)$; find the solutions approximately, e.g., using technology to graph the functions, make tables of values, or find successive approximations. Include cases where f(x) and/or g(x) are linear, polynomial, rational, absolute value, exponential, and logarithmic functions.
		\item CC.9-12.F.BF.2 Build a function that models a relationship between two quantities. Write arithmetic and geometric sequences both recursively and with an explicit formula, use them to model situations, and translate between the two forms.
		\item CC.9-12.F.BF.3 Build new functions from existing functions. Identify the effect on the graph of replacing $f(x)$ by $f(x) + k, k f(x), f(kx)$, and $f(x + k)$ for specific values of k (both positive and negative); find the value of k given the graphs. Experiment with cases and illustrate an explanation of the effects on the graph using technology. Include recognizing even and odd functions from their graphs and algebraic expressions for them.
		\item CC.9-12.F.BF.4  Find inverse functions. 
		\item CC.9-12.F.BF.4a Solve an equation of the form $f(x) = c$ for a simple function f that has an inverse and write an expression for the inverse. 
		\item CC.9-12.F.IF.4 For a function that models a relationship between two quantities, interpret key features of graphs and tables in terms of the quantities, and sketch graphs showing key features given a verbal description of the relationship. 
		\item CC.9-12.F.IF.7 Graph functions expressed symbolically and show key features of the graph, by hand in simple cases and using technology for more complicated cases.
	\end{itemize}		
	\item Supporting
	\begin{itemize}
		\item CC.9-12.F.BF.1 Write a function that describes a relationship between two quantities
		\item CC.9-12.F.BF.1b Combine standard function types using arithmetic operations. 
		\item CC.9-12.F.IF.5  Relate the domain of a function to its graph and, where applicable, to the quantitative relationship it describes.
		\item CC.9-12.F.IF.6   Calculate and interpret the average rate of change of a function (presented symbolically or as a table) over a specified interval. Estimate the rate of change from a graph.
		\item CC.9-12.F.IF.8  Write a function defined by an expression in different but equivalent forms to reveal and explain different properties of the function. 
		\item CC.9-12.F.IF.9 Compare properties of two functions each represented in a different way (algebraically, graphically, numerically in tables, or by verbal descriptions).
		\item CC.9-12.F.LE.3  Observe using graphs and tables that a quantity increasing exponentially eventually exceeds a quantity increasing linearly, quadratically, or (more generally) as a polynomial function.
	\end{itemize}
	\item Advanced
	\begin{itemize}
		\item CC.9-12.F.BF.1c (+) Compose functions. 
		\item CC.9-12.F.BF.4b (+) Verify by composition that one function is the inverse of another. 
		\item CC.9-12.F.BF.4c (+) Read values of an inverse function from a graph or a table, given that the function has an inverse.
		\item CC.9-12.F.BF.4d (+) Produce an invertible function from a non-invertible function by restricting the domain.
	\end{itemize}
	\item Advanced Supporting
	\begin{itemize}
		\item
	\end{itemize}
\end{enumerate}
\newpage
\subsection*{Standards - Quadratics}
\begin{enumerate}
	\item Essential
	\begin{itemize}
		\item CC.9-12.A.REI.4a Use the method of completing the square to transform any quadratic equation in x into an equation of the form $(x – p)^2 = q$ that has the same solutions. Derive the quadratic formula from this form. 
		\item CC.9-12.A.REI.4b Solve quadratic equations by inspection (e.g., for $x^2 = 49$), taking square roots, completing the square, the quadratic formula and factoring, as appropriate to the initial form of the equation. Recognize when the quadratic formula gives complex solutions and write them as $a \pm b\textrm{i}$ for real numbers $a$ and $b$.
		\item CC.9-12.A.REI.7 . Solve a simple system consisting of a linear equation and a quadratic equation in two variables algebraically and graphically. 
		\item CC.9-12.A.SSE.3 Choose and produce an equivalent form of an expression to reveal and explain properties of the quantity represented by the expression.
		\item CC.9-12.A.SSE.3a Factor a quadratic expression to reveal the zeros of the function it defines.
		\item CC.9-12.A.SSE.3b Complete the square in a quadratic expression to reveal the maximum or minimum value of the function it defines.
		\item CC.9-12.N.CN.2 Use the relation $\textrm{i}^2 = –1$ and the commutative, associative, and distributive properties to add, subtract, and multiply complex numbers.
		\item CC.9-12.N.CN.7  Solve quadratic equations with real coefficients that have complex solutions.
	\end{itemize}		
	\item Supporting
	\begin{itemize}
		\item CC.9-12.A.SSE.1a Interpret parts of an expression, such as terms, factors, and coefficients.*
		\item CC.9-12.F.IF.8a Use the process of factoring and completing the square in a quadratic function to show zeros, extreme values, and symmetry of the graph, and interpret these in terms of a context. 
		\item CC.9-12.N.CN.1 Know there is a complex number i such that $\textrm{i}^2 =-1$, and every complex number has the form $a + b\textrm{i}$ with $a$ and $b$ real. 
	\end{itemize}
	\item Advanced
	\begin{itemize}
		\item
	\end{itemize}
	\item Advanced Supporting
	\begin{itemize}
		\item CC.9-12.N.CN.9 (+)  Know the Fundamental Theorem of Algebra; show that it is true for quadratic polynomials.
		\item CC.9-12.N.CN.8 (+)  Extend polynomial identities to the complex numbers.

	\end{itemize}
\end{enumerate}
\newpage
\subsection*{Standards - Polynomials}
\begin{enumerate}
	\item Essential
	\begin{itemize}
		\item CC.9-12.A.APR.1  Add, subtract, and multiply polynomials.
		\item CC.9-12.A.APR.3  Identify zeros of polynomials when suitable factorizations are available.
		\item CC.9-12.A.APR.3  Use the zeros of polynomials to construct a rough graph of the function defined by the polynomial.
		\item CC.9-12.F.IF.7c Graph polynomial functions, identifying zeros when suitable factorizations are available, and showing end behavior.
	\end{itemize}		
	\item Supporting
	\begin{itemize}
		\item CC.9-12.A.APR.2 Know and apply the Remainder Theorem: For a polynomial $p(x)$ and a number $a$, the remainder on division by $x – a$ is $p(a)$, so $p(a) = 0$ if and only if $(x – a)$ is a factor of $p(x)$.
		\item CC.9-12.A.APR.4  Prove polynomial identities and use them to describe numerical relationships. For example, the polynomial identity $(x^2 + y^2)^2 = (x^2 – y^2)^2 + (2xy)^2$ can be used to generate Pythagorean triples.
		\item CC.9-12.A.SSE.1a Interpret parts of an expression, such as terms, factors, and coefficients.*
	\end{itemize}
	\item Advanced
	\begin{itemize}
		\item
	\end{itemize}
	\item Advanced Supporting
	\begin{itemize}
		\item CC.9-12.N.CN.9 (+)  Know the Fundamental Theorem of Algebra; show that it is true for quadratic polynomials.
		\item CC.9-12.N.CN.8 (+)  Extend polynomial identities to the complex numbers.		
		\item CC.9-12.A.APR.5 (+) Know and apply that the Binomial Theorem gives the expansion of $(x + y)^n$ in powers of $x$ and $y$ for a positive integer $n$, where $x$ and $y$ are any numbers, with coefficients determined for example by Pascal’s Triangle. 
	\end{itemize}
\end{enumerate}
\newpage
\subsection*{Standards - Exponential}
\begin{enumerate}
	\item Essential
	\begin{itemize}
		\item CC.9-12.A.SSE.3c Use the properties of exponents to transform expressions for exponential functions.
		\item CC.9-12.F.IF.7e Graph exponential and logarithmic functions, showing intercepts and end behavior.
		\item CC.9-12.F.LE.1  Distinguish between situations that can be modeled with linear functions and with exponential functions.
		\item CC.9-12.F.LE.1a Prove that linear functions grow by equal differences over equal intervals and that exponential functions grow by equal factors over equal intervals.
		\item CC.9-12.F.LE.1b. Recognize situations in which one quantity changes at a constant rate per unit interval relative to another.
		\item CC.9-12.F.LE.1c Recognize situations in which a quantity grows or decays by a constant percent rate per unit interval relative to another.
		\item CC.9-12.F.LE.4 For exponential models, express as a logarithm the solution to$ ab^{ct} = d$ where $a, c$, and $d$ are numbers and the base $b$ is 2, 10, or e; evaluate the logarithm using technology.
		\item CC.9-12.F.LE.5  Interpret the parameters in a linear or exponential function in terms of a context.
	\end{itemize}		
	\item Supporting
	\begin{itemize}
		\item CC.9-12.F.IF.8b Use the properties of exponents to interpret expressions for exponential functions. 
		\item CC.9-12.F.LE.2  Construct linear and exponential functions, including arithmetic and geometric sequences, given a graph, a description of a relationship, or two input-output pairs (include reading these from a table).
	\end{itemize}
	\item Advanced
	\begin{itemize}
		\item
	\end{itemize}
	\item Advanced Supporting
	\begin{itemize}
		\item CC.9-12.F.BF.5 (+) Understand the inverse relationship between exponents and logarithms and use this relationship to solve problems involving logarithms and exponents.
	\end{itemize}
\end{enumerate}
\newpage
\subsection*{Standards - Rational}
\begin{enumerate}
	\item Essential
	\begin{itemize}
		\item CC.9-12.A.REI.2 Solve simple rational equations in one variable, and give examples showing how extraneous solutions may arise.
	\end{itemize}		
	\item Supporting
	\begin{itemize}
		\item CC.9-12.A.APR.6 Rewrite rational expressions. Rewrite simple rational expressions in different forms; write $a(x)/b(x)$ in the form $q(x) + r(x)/b(x)$, where $a(x), b(x), q(x)$, and $r(x)$ are polynomials with the degree of $r(x)$ less than the degree of $b(x)$, using inspection, long division, or, for the more complicated examples, a computer algebra system.
	\end{itemize}
	\item Advanced
	\begin{itemize}
		\item CC.9-12.F.IF.7d (+) Graph rational functions, identifying zeros and asymptotes when suitable factorizations are available, and showing end behavior. 
	\end{itemize}
	\item Advanced Supporting
	\begin{itemize}
		\item CC.9-12.A.APR.7 (+)  Understand that rational expressions form a system analogous to the rational numbers, closed under addition, subtraction, multiplication, and division by a nonzero rational expression; add, subtract, multiply, and divide rational expressions.
	\end{itemize}
\end{enumerate}
\newpage
\subsection*{Standards - Radical}
\begin{enumerate}
	\item Essential
	\begin{itemize}
		\item CC.9-12.A.REI.2 Solve radical equations in one variable, and give examples showing how extraneous solutions may arise.
		\item CC.9-12.N.RN.2  Rewrite expressions involving radicals and rational exponents using the properties of exponents.
		\item CC.9-12.N.RN.1 Explain how the definition of the meaning of rational exponents follows from extending the properties of integer exponents to those values, allowing for a notation for radicals in terms of rational exponents. 
	\end{itemize}		
	\item Supporting
	\begin{itemize}
		\item
	\end{itemize}
	\item Advanced
	\begin{itemize}
		\item CC.9-12.F.BF.4d (+) Produce an invertible function from a non-invertible function by restricting the domain.
	\end{itemize}
	\item Advanced Supporting
	\begin{itemize}
		\item
	\end{itemize}
\end{enumerate}
\newpage
\subsection*{Standards - Trigonometry}
\begin{enumerate}
	\item Essential
	\begin{itemize}
		\item CC.9-12.F.IF.7e Graph trigonometric functions, showing period, midline, and amplitude.
		\item CC.9-12.F.TF.2  Explain how the unit circle in the coordinate plane enables the extension of trigonometric functions to all real numbers, interpreted as radian measures of angles traversed counterclockwise around the unit circle.
		\item CC.9-12.F.TF.5  Choose trigonometric functions to model periodic phenomena with specified amplitude, frequency, and midline.
		\item CC.9-12.F.TF.8  Prove the Pythagorean identity $(\sin A)^2 + (\cos A)^2 = 1$ and use it to find $\sin A, \cos A$, or $\tan A$, given $\sin A, \cos A$, or$\ tan A$, and the quadrant of the angle.
		\item CC.9-12.G.GPE.4 Use coordinates to prove simple geometric theorems algebraically. 
		\item CC.9-12.G.SRT.8 . Use trigonometric ratios and the Pythagorean Theorem to solve right triangles in applied problems.
	\end{itemize}		
	\item Supporting
	\begin{itemize}
		\item CC.9-12.F.TF.1 Understand radian measure of an angle as the length of the arc on the unit circle subtended by the angle.
		\item CC.9-12.G.C.5  Derive using similarity the fact that the length of the arc intercepted by an angle is proportional to the radius, and define the radian measure of the angle as the constant of proportionality; derive the formula for the area of a sector.
		\item CC.9-12.G.SRT.6 Understand that by similarity, side ratios in right triangles are properties of the angles in the triangle, leading to definitions of trigonometric ratios for acute angles.
		\item CC.9-12.G.SRT.7  Explain and use the relationship between the sine and cosine of complementary angles.
	\end{itemize}
	\item Advanced
	\begin{itemize}
		\item CC.9-12.F.BF.4d (+) Produce an invertible function from a non-invertible function by restricting the domain.
	\end{itemize}
	\item Advanced Supporting
	\begin{itemize}
		\item CC.9-12.F.TF.3 (+)  Use special triangles to determine geometrically the values of sine, cosine, tangent for $\pi3, \pi/4$ and $\pi/6$, and use the unit circle to express the values of sine, cosine, and tangent for $\pi - x, \pi + x$, and $2\pi - x$ in terms of their values for $x$, where $x$ is any real number.
		\item CC.9-12.F.TF.4 (+)  Use the unit circle to explain symmetry (odd and even) and periodicity of trigonometric functions.
		\item CC.9-12.F.TF.6 (+)  Understand that restricting a trigonometric function to a domain on which it is always increasing or always decreasing allows its inverse to be constructed.
		\item CC.9-12.F.TF.7 (+) Use inverse functions to solve trigonometric equations that arise in modeling contexts; evaluate the solutions using technology, and interpret them in terms of the context.*
		\item CC.9-12.F.TF.9 (+)  Prove the addition and subtraction formulas for sine, cosine, and tangent and use them to solve problems.
		\item CC.9-12.G.SRT.9 (+)   Derive the formula $A = (1/2)ab \sin(C)$ for the area of a triangle by drawing an auxiliary line from a vertex perpendicular to the opposite side.
		\item CC.9-12.G.SRT.10 (+) Prove the Laws of Sines and Cosines and use them to solve problems.
		\item CC.9-12.G.SRT.11 (+)  Understand and apply the Law of Sines and the Law of Cosines to find unknown measurements in right and non-right triangles (e.g., surveying problems, resultant forces).
		\item CC.9-12.N.CN.3 (+)  Find the conjugate of a complex number; use conjugates to find moduli and quotients of complex numbers.
		\item CC.9-12.N.CN.4 (+) Represent complex numbers on the complex plane in rectangular and polar form (including real and imaginary numbers), and explain why the rectangular and polar forms of a given complex number represent the same number.
		\item CC.9-12.N.CN.5 (+) Represent addition, subtraction, multiplication, and conjugation of complex numbers geometrically on the complex plane; use properties of this representation for computation. 
		\item CC.9-12.N.CN.6 (+)  Calculate the distance between numbers in the complex plane as the modulus of the difference, and the midpoint of a segment as the average of the numbers at its endpoints.
	\end{itemize}
\end{enumerate}
\newpage
\subsection*{Standards - Vectors and Matrices}
\begin{enumerate}
	\item Essential
	\begin{itemize}
		\item
	\end{itemize}		
	\item Supporting
	\begin{itemize}
		\item
	\end{itemize}
	\item Advanced
	\begin{itemize}
		\item
	\end{itemize}
	\item Advanced Supporting
	\begin{itemize}
		\item CC.9-12.A.REI.8 (+)  Represent a system of linear equations as a single matrix equation in a vector variable.
		\item CC.9-12.A.REI.9 (+)  Find the inverse of a matrix if it exists and use it to solve systems of linear equations (using technology for matrices of dimension $3 \times 3$ or greater).
		\item CC.9-12.N.VM.1 (+) Recognize vector quantities as having both magnitude and direction. Represent vector quantities by directed line segments, and use appropriate symbols for vectors and their magnitudes (e.g., \textbf{$v$}, $|$\textbf{$v$}$|$, $||$\textbf{$v$}$||$, $\vec{v}$).
		\item CC.9-12.N.VM.2 (+)  Find the components of a vector by subtracting the coordinates of an initial point from the coordinates of a terminal point.
		\item CC.9-12.N.VM.3 (+)   Solve problems involving velocity and other quantities that can be represented by vectors.
		\item CC.9-12.N.VM.4 (+)  Add and subtract vectors.
		\item CC.9-12.N.VM.4a (+) Add vectors end-to-end, component-wise, and by the parallelogram rule. Understand that the magnitude of a sum of two vectors is typically not the sum of the magnitudes.
		\item CC.9-12.N.VM.4b (+) Given two vectors in magnitude and direction form, determine the magnitude and direction of their sum. 
		\item CC.9-12.N.VM.4c (+) Understand vector subtraction $v – w$ as $v + (–w)$, where $(–w)$ is the additive inverse of $w$, with the same magnitude as w and pointing in the opposite direction. Represent vector subtraction graphically by connecting the tips in the appropriate order, and perform vector subtraction component-wise.
		\item CC.9-12.N.VM.5 (+)  Multiply a vector by a scalar.
		\item CC.9-12.N.VM.5a (+) Represent scalar multiplication graphically by scaling vectors and possibly reversing their direction; perform scalar multiplication component-wise, e.g., as $c(\vec{v}_x, \vec{v}_y) = (c\vec{v}_x, c\vec{v}_y)$.
		\item CC.9-12.N.VM.5b (+) Compute the magnitude of a scalar multiple $c\vec{v}$ using $||c\vec{v}|| =|c|\vec{v}$. Compute the direction of $c\vec{v}$ knowing that when $|c|\vec{v} \neq 0$, the direction of $c\vec{v}$ is either along $\vec{v}$ (for $c > 0$) or against $\vec{v}$ (for $c < 0$).
		\item CC.9-12.N.VM.6 (+)  Use matrices to represent and manipulate data, e.g., to represent payoffs or incidence relationships in a network. 
		\item CC.9-12.N.VM.7 (+)  Multiply matrices by scalars to produce new matrices, e.g., as when all of the payoffs in a game are doubled. 
		\item CC.9-12.N.VM.8 (+)  Add, subtract, and multiply matrices of appropriate dimensions.
		\item CC.9-12.N.VM.9 (+) Understand that, unlike multiplication of numbers, matrix multiplication for square matrices is not a commutative operation, but still satisfies the associative and distributive properties. 
		\item CC.9-12.N.VM.10 (+) Understand that the zero and identity matrices play a role in matrix addition and multiplication similar to the role of 0 and 1 in the real numbers. The determinant of a square matrix is nonzero if and only if the matrix has a multiplicative inverse. 
		\item CC.9-12.N.VM.11 (+) Multiply a vector (regarded as a matrix with one column) by a matrix of suitable dimensions to produce another vector. Work with matrices as transformations of vectors.
		\item CC.9-12.N.VM.12 (+) Work with $2 \times 2$ matrices as transformations of the plane, and interpret the absolute value of the determinant in terms of area. 
	\end{itemize}
\end{enumerate}
\newpage
\subsection*{Standards - Probability and Statistics}
\begin{enumerate}
	\item Essential
	\begin{itemize}
		\item CC.9-12.S.ID.4  Use the mean and standard deviation of a data set to fit it to a normal distribution and to estimate population percentages. Recognize that there are data sets for which such a procedure is not appropriate. Use calculators, spreadsheets, and tables to estimate areas under the normal curve.
		\item CC.9-12.S.ID.6  Represent data on two quantitative variables on a scatter plot, and describe how the variables are related.
		\item CC.9-12.S.ID.6a Fit a function to the data; use functions fitted to data to solve problems in the context of the data. 
	\end{itemize}		
	\item Supporting
	\begin{itemize}
		\item CC.9-12.S.IC.1  Understand statistics as a process for making inferences about population parameters based on a random sample from that population.
		\item CC.9-12.S.IC.2  Decide if a specified model is consistent with results from a given data-generating process, e.g., using simulation. 
		\item CC.9-12.S.IC.3  Recognize the purposes of and differences among sample surveys, experiments, and observational studies; explain how randomization relates to each.
		\item CC.9-12.S.IC.4  Use data from a sample survey to estimate a population mean or proportion; develop a margin of error through the use of simulation models for random sampling.
		\item CC.9-12.S.IC.5  Use data from a randomized experiment to compare two treatments; use simulations to decide if differences between parameters are significant.
		\item CC.9-12.S.IC.6  Evaluate reports based on data.
		\item CC.9-12.S.ID.3 Interpret differences in shape, center, and spread in the context of the data sets, accounting for possible effects of extreme data points (outliers).
		\item CC.9-12.S.ID.8  Compute (using technology) and interpret the correlation coefficient of a linear fit.
		\item CC.9-12.S.ID.9  Distinguish between correlation and causation.
		\item CC.9-12.A.APR.5 (+) Know and apply that the Binomial Theorem gives the expansion of $(x + y)^n$ in powers of $x$ and $y$ for a positive integer $n$, where $x$ and $y$ are any numbers, with coefficients determined for example by Pascal’s Triangle. 
	\end{itemize}
	\item Advanced
	\begin{itemize}
		\item
	\end{itemize}
	\item Advanced Supporting
	\begin{itemize}
		\item CC.9-12.S.CP.8 (+)  Apply the general Multiplication Rule in a uniform probability model, $\textrm{P}(A \textrm{and} B) = [\textrm{P}(A)]\times[\textrm{P}(B|A)] =[\textrm{P}(B)]\times[\textrm{P}(A|B)]$, and interpret the answer in terms of the model.
		\item CC.9-12.S.CP.9 (+) Use permutations and combinations to compute probabilities of compound events and solve problems.
		\item CC.9-12.S.MD.1 (+) Define a random variable for a quantity of interest by assigning a numerical value to each event in a sample space; graph the corresponding probability distribution using the same graphical displays as for data distributions.
		\item CC.9-12.S.MD.2 (+) Calculate the expected value of a random variable; interpret it as the mean of the probability distribution.
		\item CC.9-12.S.MD.3 (+) Develop a probability distribution for a random variable defined for a sample space in which theoretical probabilities can be calculated; find the expected value. 
		\item CC.9-12.S.MD.4 (+)  Develop a probability distribution for a random variable defined for a sample space in which probabilities are assigned empirically; find the expected value. 
		\item CC.9-12.S.MD.5 (+)  Weigh the possible outcomes of a decision by assigning probabilities to payoff values and finding expected values.
		\item CC.9-12.S.MD.5a (+) Find the expected payoff for a game of chance. 
		\item CC.9-12.S.MD.5b (+)  Evaluate and compare strategies on the basis of expected values. 
		\item CC.9-12.S.MD.6 (+)  Use probabilities to make fair decisions (e.g., drawing by lots, using a random number generator).
		\item CC.9-12.S.MD.7 (+)  Analyze decisions and strategies using probability concepts (e.g., product testing, medical testing, pulling a hockey goalie at the end of a game).
	\end{itemize}
\end{enumerate}
\newpage
\subsection*{Standards - Sequences and Series}
\begin{enumerate}
	\item Essential
	\begin{itemize}
		\item
	\end{itemize}		
	\item Supporting
	\begin{itemize}
		\item CC.9-12.A.SSE.4  Derive the formula for the sum of a finite geometric series (when the common ratio is not 1), and use the formula to solve problems. 
		\item CC.9-12.F.BF.1a Determine an explicit expression, a recursive process, or steps for calculation from a context.
		\item CC.9-12.F.IF.3 . Recognize that sequences are functions, sometimes defined recursively, whose domain is a subset of the integers. 
		\item CC.9-12.F.LE.2  Construct linear and exponential functions, including arithmetic and geometric sequences, given a graph, a description of a relationship, or two input-output pairs (include reading these from a table).
	\end{itemize}
	\item Advanced
	\begin{itemize}
		\item
	\end{itemize}
	\item Advanced Supporting
	\begin{itemize}
		\item
	\end{itemize}
\end{enumerate}
\newpage
\subsection*{Standards - Conics}
\begin{enumerate}
	\item Essential
	\begin{itemize}
		\item
	\end{itemize}		
	\item Supporting
	\begin{itemize}
		\item CC.9-12.G.GPE.1  Derive the equation of a circle of given center and radius using the Pythagorean Theorem; complete the square to find the center and radius of a circle given by an equation.
		\item CC.9-12.G.GPE.2 Derive the equation of a parabola given a focus and directrix.
	\end{itemize}
	\item Advanced
	\begin{itemize}
		\item
	\end{itemize}
	\item Advanced Supporting
	\begin{itemize}
		\item CC.9-12.G.GPE.3 (+)  Derive the equations of ellipses and hyperbolas given the foci, using the fact that the sum or difference of distances from the foci is constant.
	\end{itemize}
\end{enumerate}

%%%%%%%%%%%%%%%%%%%%%%%%%%%%%%%%%%%%%%%%%%%%%%%%%%%%%%%%%%%%%%%%%%%%%%%%%
%%%%%%%%%%%%%    LEARNING TARGETS
%%%%%%%%%%%%%%%%%%%%%%%%%%%%%%%%%%%%%%%%%%%%%%%%%%%%%%%%%%%%%%%%%%%%%%%%%
\newpage
\noindent
\textbf{Learning Targets}
\begin{enumerate}
	\item [2.] D Level Learning Targets
	\item [3.] C Level Learning Targets
	\item [4.] B Level Learning Targets
	\item [5.] A Level Learning Targets
\end{enumerate}
\section{Targets - Functions}
\begin{enumerate}
	\item Use technology to find the solutions to of the equation $f(x) = g(x)$. Include cases where f(x) and/or g(x) are linear, polynomial, rational, absolute value, exponential, and logarithmic functions. [2] [CC.9-12.A.REI.11]
	\item Sketch the graphs of functions using transformations of their parent function. [3] [CC.9-12.F.BF.3]
	\item Be able to determine from a graph whether a function is odd, even, or neither. [3] [CC.9-12.F.BF.3]
	\item Find the inverse of a function. [3] [CC.9-12.F.BF.4]
	\item Combine functions using arithmetic operations. [3] [CC.9-12.F.BF.1b]
	\item Determine the domain of a function from its graph. [3] [CC.9-12.F.IF.5]
	\item Write functions that model relationships between two quantities. [3] [CC.9-12.F.BF.1; CC.9-12.F.BF.2]
	\item Interpret key features of graph of functions. [3] [CC.9-12.F.IF.4]
	\item Be able to algebraically determine whether a function is odd, even, or neither. [4] [CC.9-12.F.BF.3]
	\item Determine the domain of a function algebraically. [4] [CC.9-12.F.IF.5]
	\item Graph the inverse of a function given the graph of a function. [4] [CC.9-12.F.BF.4c]
	\item Perform function composition. [5] [CC.9-12.F.BF.1c]
	\item Verify two functions are inverses by composition. [5] [CC.9-12.F.BF.4b]
\end{enumerate}
\newpage
\section{Targets - Quadratics}
\begin{enumerate}
	\item Interpret parts of an expression, such as terms, factors, and coefficients. [2] [CC.9-12.A.SSE.1a ]
	\item Determine the minumim or maximum value of a quadratic function written in vertex form. [2] [CC.9-12.A.SSE.3b]
	\item Solve quadratic equations by taking square roots. [2] [CC.9-12.A.REI.4b]
	\item Solve quadratic equations by using the quadratic formula. [2] [CC.9-12.A.REI.4b]
	\item Solve a simple system consisting of a linear equation and a quadratic equation in two variables graphically (with technology). [2] [CC.9-12.A.REI.7 ]
	\item Know there is a complex number i such that $\textrm{i}^2 =-1$, and every complex number has the form $a + b\textrm{i}$ with $a$ and $b$ real.	 [2] [CC.9-12.N.CN.1 ]
	\item Factor quadratic expressions. [3] [CC.9-12.A.SSE.3a] 
	\item Solve quadratic equations by factoring. [3] [CC.9-12.A.REI.4b]
	\item Solve a simple system consisting of a linear equation and a quadratic equation in two variables graphically (without technology). [3] [CC.9-12.A.REI.7 ] 
	\item Use the process of factoring and completing the square in a quadratic function to show zeros, extreme values, and symmetry of the graph, and interpret these in terms of a context. [4] [CC.9-12.F.IF.8a]
	\item Add and subtract complex numbers. [3] [CC.9-12.N.CN.2 ]
	\item Complete the square to transform any quadratic function into vertex form. [4] [CC.9-12.A.REI.4a] [CC.9-12.A.SSE.3b]
	\item Solve quadratic equations by completing the square. [4] [CC.9-12.A.REI.4b]
	\item Multiply complex numbers. [4] [CC.9-12.N.CN.2 ]
	\item Solve a simple system consisting of a linear equation and a quadratic equation in two variables algebraically. [4] [CC.9-12.A.REI.7 ]	
	\item Derive the quadratic formula by completing the square on the general form of a quadratic equation. [5] [CC.9-12.A.REI.4a]	
\end{enumerate}
\newpage
\section{Targets - Polynomials}
\begin{enumerate}
	\item Add and subtract polynomials. [2] [CC.9-12.A.APR.1]
	\item Determine the degree, leading coefficient, constant term, and maximum number of turning points for a polynomial function. [2] [CC.9-12.A.SSE.1a]
	\item Use the zeros of polynomials to construct a rough graph of the function defined by the polynomial. [2] [CC.9-12.A.APR.3]
	\item Multiply polynomials. [3] [CC.9-12.A.APR.1]
	\item Divide polynomials by linear factors using synthetic division or polynomial long division. [3] [CC.9-12.A.APR.2]
	\item Identify zeros of polynomials when suitable factorizations are available. [3] [CC.9-12.A.APR.3]
	\item Graph polynomial functions, identifying zeros when suitable factorizations are available, and showing end behavior.[4] [CC.9-12.F.IF.7c]
	\item Use the remainder theorem to assist in factoring polynomials. [4] [CC.9-12.A.APR.2]
	\item Know and apply that the Binomial Theorem gives the expansion of $(x + y)^n$ in powers of $x$ and $y$ for a positive integer $n$, where $x$ and $y$ are any numbers [4] [CC.9-12.A.APR.5 ]
\end{enumerate}
\newpage
\section{Targets - Exponential}
\begin{enumerate}
	\item Recognize situations in which one quantity changes at a constant rate per unit interval relative to another. [2] [CC.9-12.F.LE.1b]
	\item Recognize situations in which a quantity grows or decays by a constant percent rate per unit interval relative to another. [2] [CC.9-12.F.LE.1c]
	\item Use the properties of exponents to transform expressions for exponential functions. [2] [CC.9-12.A.SSE.3c]
	\item Distinguish between situations that can be modeled with linear functions and with exponential functions. [3] [CC.9-12.F.LE.1 ]
	\item Interpret the parameters in a linear or exponential function in terms of a context. [3] [CC.9-12.F.LE.5]
	\item Use the properties of exponents to interpret expressions for exponential functions. [3] [CC.9-12.F.IF.8b]
	\item Prove that linear functions grow by equal differences over equal intervals and that exponential functions grow by equal factors over equal intervals. [3] [CC.9-12.F.LE.1a]
	\item Graph exponential functions showing intercepts and end behavior. [3] [CC.9-12.F.IF.7e]
	\item Graph logarithmic functions showing intercepts and end behavior. [3] [CC.9-12.F.IF.7e]
	\item For exponential models, express as a logarithm the solution to$ ab^{ct} = d$ where $a, c$, and $d$ are numbers and the base $b$ is 2, 10, or e; evaluate the logarithm using technology. [4] [CC.9-12.F.LE.4]
	\item Construct linear and exponential functions, including arithmetic and geometric sequences, given a graph, a description of a relationship, or two input-output pairs (include reading these from a table). [4] [CC.9-12.F.LE.2]
	\item Understand the inverse relationship between exponents and logarithms and use this relationship to solve problems involving logarithms and exponents.[5] [CC.9-12.F.BF.5]
\end{enumerate}
\newpage
\section*{Targets - Rational}
\begin{enumerate}
	\item Using the provided procedure, graph rational functions without holes or oblique asymptotes that are written in factored form. [3] [CC.9-12.F.IF.7d]
	\item Add, subtract, multiply, and divide rational expressions. [4] [CC.9-12.A.APR.7]
	\item Solve simple rational equations in one variable, and give examples showing how extraneous solutions may arise. [4] [CC.9-12.A.REI.2]
	\item Rewrite simple rational expressions in different forms; write $a(x)/b(x)$ in the form $q(x) + r(x)/b(x)$, where $a(x), b(x), q(x)$, and $r(x)$ are polynomials with the degree of $r(x)$ less than the degree of $b(x)$. [4] [CC.9-12.A.APR.6 ]
	\item Graph rational functions, identifying zeros and asymptotes (horizontal and vertical) when suitable factorizations are available, and showing end behavior. [4] [CC.9-12.F.IF.7d]
	\item Graph rational functions, identifying zeros, holes, and asymptotes (oblique and vertical) when suitable factorizations are available, and showing end behavior. [5] [CC.9-12.F.IF.7d]
\end{enumerate}
\newpage
\section*{Targets - Radical}
\begin{enumerate}
	\item Convert expressions written as radicals to rational exponents. [2] [CC.9-12.N.RN.1]
	\item Rewrite expressions involving radicals and rational exponents using the properties of exponents. [2] [CC.9-12.N.RN.2]
	\item Solve radical equations in one variable[3] [CC.9-12.A.REI.2 ]
\end{enumerate}
\newpage
\section*{Targets - Trigonometry}
\begin{enumerate}
	\item Add and subtract vectors graphically. [2] [CC.9-12.N.VM.4 ]
	\item Add and subtract vectors algebraically. [2] [CC.9-12.N.VM.4 ]
	\item Perform scalar multiplication graphically.  [2] [CC.9-12.N.VM.5a]
	\item Perform scalar multiplication algebraically.  [2] [CC.9-12.N.VM.5]
	\item Find the components of a vector given its initial and terminal points. [2] [CC.9-12.N.VM.2]
	\item Know the definitions of trigonometric ratios for acute angles. [2] [CC.9-12.G.SRT.6]
	\item Explain and use the relationship between the sine and cosine of complementary angles. [2] [CC.9-12.G.SRT.7]
	\item Convert from degrees to radians and from radians to degrees. [2] [CC.9-12.F.TF.1]
	\item Derive the equation of a circle of given center and radius using the Pythagorean Theorem. [3] [CC.9-12.G.GPE.1]
	\item Use trigonometric ratios and the Pythagorean Theorem to solve right triangles in applied problems. [3] [CC.9-12.G.SRT.8]
	\item Know and be able to use the unit circle definition of the trigonometric functions. [3] [CC.9-12.F.TF.2]
	\item Use inverse functions to solve trigonometric equations that arise in modeling contexts; evaluate the solutions using technology, and interpret them in terms of the context. [3] [CC.9-12.F.TF.7 ]
	\item Graph trigonometric functions, showing period, midline, and amplitude. [3] [CC.9-12.F.IF.7e]
	\item Use the addition and subtraction formulas for sine, cosine, and tangent to solve problems.[3] [CC.9-12.F.TF.9]
	\item Determine the area of a sector of a cicrle. [3] [CC.9-12.G.C.5]
	\item Given two vectors in magnitude and direction form, determine the magnitude and direction of their sum. [3] [CC.9-12.N.VM.4b]
	\item Use special triangles to determine geometrically the values of sine, cosine, tangent for $\pi3, \pi/4$ and $\pi/6$, and use the unit circle to express the values of sine, cosine, and tangent for $\pi - x, \pi + x$, and $2\pi - x$ in terms of their values for $x$, where $x$ is any real number. [4] [CC.9-12.F.TF.3 ]
	\item Choose trigonometric functions to model periodic phenomena with specified amplitude, frequency, and midline. [4] [CC.9-12.F.TF.5]
	\item Understand and apply the Law of Sines and the Law of Cosines to find unknown measurements in right and non-right triangles (e.g., surveying problems, resultant forces). [4] [CC.9-12.G.SRT.11]
	\item Find the conjugate of a complex number; use conjugates to find moduli and quotients of complex numbers. [4] [CC.9-12.N.CN.3]
	\item Solve problems involving velocity and other quantities that can be represented by vectors. [4] [CC.9-12.N.VM.3]
	\item Simplify trigonometric expressions given the fundamental trigonometric identities. [4]  [CC.9-12.F.TF.9]
	\item Represent complex numbers on the complex plane in rectangular and polar form (including real and imaginary numbers), and explain why the rectangular and polar forms of a given complex number represent the same number. [5] [CC.9-12.N.CN.4 ]
	\item Represent addition, subtraction, multiplication, and conjugation of complex numbers geometrically on the complex plane; use properties of this representation for computation. [5] [CC.9-12.N.CN.5]
	\item Calculate the distance between numbers in the complex plane as the modulus of the difference, and the midpoint of a segment as the average of the numbers at its endpoints. [5] [CC.9-12.N.CN.6]
	\item Prove the addition and subtraction formulas for sine, cosine, and tangent and use them to solve problems.[5] [CC.9-12.F.TF.9]
	\item Prove the Pythagorean identies and use them to find $\sin A, \cos A$, or $\tan A$, given $\sin A, \cos A$, or$\ tan A$, and the quadrant of the angle. [5] [CC.9-12.F.TF.8]
	\item Know and use Heron's formula and the formula $A = (1/2)ab \sin(C)$ to find the area of a triangle. [5] [ [CC.9-12.G.SRT.9]
	\item Prove trigonometric identities using the fundamental trigonometric definitions and identities. [5] [CC.9-12.F.TF.9]
\end{enumerate}
\newpage
\section*{Targets - Probability and Statistics}
\begin{enumerate}

	\item CC.9-12.S.ID.3 Interpret differences in shape, center, and spread in the context of the data sets, accounting for possible effects of extreme data points (outliers).

	\item CC.9-12.S.ID.6  Represent data on two quantitative variables on a scatter plot, and describe how the variables are related.
	\item CC.9-12.S.ID.4  Use the mean and standard deviation of a data set to fit it to a normal distribution and to estimate population percentages. Recognize that there are data sets for which such a procedure is not appropriate. Use calculators, spreadsheets, and tables to estimate areas under the normal curve.
	\item CC.9-12.S.CP.8 (+)  Apply the general Multiplication Rule in a uniform probability model, $\textrm{P}(A \textrm{and} B) = [\textrm{P}(A)]\times[\textrm{P}(B|A)] =[\textrm{P}(B)]\times[\textrm{P}(A|B)]$, and interpret the answer in terms of the model.
	\item CC.9-12.S.CP.9 (+) Use permutations and combinations to compute probabilities of compound events and solve problems.


	\item CC.9-12.S.IC.3  Recognize the purposes of and differences among sample surveys, experiments, and observational studies; explain how randomization relates to each.
	\item CC.9-12.S.IC.4  Use data from a sample survey to estimate a population mean or proportion; develop a margin of error through the use of simulation models for random sampling.
	\item CC.9-12.S.IC.5  Use data from a randomized experiment to compare two treatments; use simulations to decide if differences between parameters are significant.


	\item CC.9-12.S.ID.6a Fit a function to the data; use functions fitted to data to solve problems in the context of the data. 
	\item CC.9-12.S.IC.1  Understand statistics as a process for making inferences about population parameters based on a random sample from that population.
	\item CC.9-12.S.IC.2  Decide if a specified model is consistent with results from a given data-generating process, e.g., using simulation. 
	\item CC.9-12.S.IC.6  Evaluate reports based on data.
	\item CC.9-12.S.ID.8  Compute (using technology) and interpret the correlation coefficient of a linear fit.
	\item CC.9-12.S.ID.9  Distinguish between correlation and causation.
	\item CC.9-12.S.MD.1 (+) Define a random variable for a quantity of interest by assigning a numerical value to each event in a sample space; graph the corresponding probability distribution using the same graphical displays as for data distributions.
	\item CC.9-12.S.MD.2 (+) Calculate the expected value of a random variable; interpret it as the mean of the probability distribution.
	\item CC.9-12.S.MD.3 (+) Develop a probability distribution for a random variable defined for a sample space in which theoretical probabilities can be calculated; find the expected value. 
	\item CC.9-12.S.MD.4 (+)  Develop a probability distribution for a random variable defined for a sample space in which probabilities are assigned empirically; find the expected value. 
	\item CC.9-12.S.MD.5 (+)  Weigh the possible outcomes of a decision by assigning probabilities to payoff values and finding expected values.

	\item CC.9-12.S.MD.5a (+) Find the expected payoff for a game of chance. 
	\item CC.9-12.S.MD.5b (+)  Evaluate and compare strategies on the basis of expected values. 
	\item CC.9-12.S.MD.6 (+)  Use probabilities to make fair decisions (e.g., drawing by lots, using a random number generator).
	\item CC.9-12.S.MD.7 (+)  Analyze decisions and strategies using probability concepts (e.g., product testing, medical testing, pulling a hockey goalie at the end of a game).
\end{enumerate}
\newpage
\section*{Targets - Sequences and Series}
\begin{enumerate}
	\item CC.9-12.A.SSE.4  Derive the formula for the sum of a finite geometric series (when the common ratio is not 1), and use the formula to solve problems. 
	\item CC.9-12.F.BF.1a Determine an explicit expression, a recursive process, or steps for calculation from a context.
	\item CC.9-12.F.IF.3 . Recognize that sequences are functions, sometimes defined recursively, whose domain is a subset of the integers. 
	\item CC.9-12.F.LE.2  Construct linear and exponential functions, including arithmetic and geometric sequences, given a graph, a description of a relationship, or two input-output pairs (include reading these from a table).
\end{enumerate}
\newpage
\subsection*{Targets - Conics}
\begin{enumerate}
	\item CC.9-12.G.GPE.1 Derive the equation of a circle of given center and radius using the Pythagorean Theorem; complete the square to find the center and radius of a circle given by an equation.
	\item CC.9-12.G.GPE.2 Derive the equation of a parabola given a focus and directrix.
	\item CC.9-12.G.GPE.3 Derive the equations of ellipses and hyperbolas given the foci, using the fact that the sum or difference of distances from the foci is constant.
\end{enumerate}
\end{document}
